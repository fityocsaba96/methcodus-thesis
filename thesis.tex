\UseRawInputEncoding
\documentclass{elteikthesis}

\title{Methcodus}
\date{2019}
\author{Fityó Csaba}
\degree{programtervező informatikus BSc}
\supervisor{Nagy Sára}
\affiliation{mesteroktató}
\extsupervisor{Czeller Ildikó}
\extaffiliation{adattudós}
\university{Eötvös Loránd Tudományegyetem}
\faculty{Informatikai Kar}
\department{Algoritmusok és Alkalmazásaik}
\departmentSecondLine{Tanszék}
\city{Budapest}
\logo{elte-crest}

\begin{document}
	\documentlang{magyar}
	\maketitle
	\tableofcontents

	\chapter{Bevezetés}
	\section{Motiváció}
	TODO (témaválasztás indoklása)

	\section{Feladatleírás}
	TODO (röviden, közérthetően)

	\chapter{Felhasználói dokumentáció}
	\section{Felhasználási igények}
	TODO (mire való, kik mikor mire használják, összehasonlítás Codewarssal és hagyományos fejlesztőkörnyezettel)
	
	\section{Használati követelmények}
	TODO (rendszerkövetelmények, tudásbeli követelmények, elérési link)
	
	\section{Alkalmazás funkciói}
	TODO (minden funkció leírása képernyőképekkel, külön regisztrált és nem regisztrált felhasználóra, külön helyes és hibás használat)
	
	\section{Szoftverfejlesztési módszerek}
	TODO (tdd, párprogramozás bemutatása)

	\chapter{Fejlesztői dokumentáció}
	\section{Probléma specifikáció}
	TODO (célok meghatározása fejlesztői szemmel)

	\section{Tervezés}
	\subsection{Használati esetek}
	TODO (use case diagram)
	
	\subsection{Felület terv}
	TODO (mockupok)
	
	\subsection{Architektúra terv}
	TODO (rövid leírás a technikai megvalósításról)
	
	\section{Szerver architektúra}
	\subsection{Authentikáció}
	TODO (folyamata)
	
	\subsection{Adatbázis}
	TODO (modellek leírása és ábra, kapcsolatok)
	
	\subsection{Végpontok}
	TODO (leírásuk, működésük, kérés és válasz tartalma, hibalehetőségek)
	
	\subsection{Felhasznált technológiák}
	TODO (mit tud, mire és hogy használom: rest, node, typescript, nest, mongodb, docker, ubuntu, java, javascript, junit, jasmine, babel, core-js, websocket, socket.io, passport, jwt, ...)
	
	\subsection{Kód felépítése}
	TODO (osztálydiagram, mappaszerkezet, rétegek)
	
	\section{Kliens architektúra}
	\subsection{Authentikáció}
	TODO (folyamata)
	
	\subsection{Képernyők}
	TODO (leírásuk, komponenseik)
	
	\subsection{Komponensek}
	TODO (leírásuk, működésük, hibalehetőségek)
	
	\subsection{Felhasznált technológiák}
	TODO (mit tud, mire és hogy használom: angular, typescript, redux, spa, emarsys design system, ...)
	
	\subsection{Kód felépítése}
	TODO (komponensdiagram, mappaszerkezet, rétegek)
	
	\section{Komponensek közti kommunikáció}
	TODO (szerver, kliens és adatbázis összekötés, szekvenciadiagram egy összetettebb funkcióra)
	
	\section{Fejlesztőkörnyezet}
	TODO (rendszerkövetelmények, beállítás, futtatás)
	
	\section{Fejlesztési folyamat}
	\subsection{Csomagkezelés}
	TODO (npm)
	
	\subsection{Verziókövetés}
	TODO (git)
	
	\subsection{Continuous Integration}
	TODO (codeship: lint, teszt, deploy)
	
	\subsection{Continuous Delivery}
	TODO (heroku)
	
	\section{Tesztelés}
	\subsection{Egységtesztek}
	TODO (lib leírás, tesztelés eredménye)
	
	\subsection{Manuális tesztek}
	TODO (minden funkció minden kimenetelére given-when-then leírás)
	
	\section{Továbbfejlesztési lehetőségek}
	TODO (clean code, élő menet, párprogramozásos élő menet)

	\begin{thebibliography}{99}
		\addcontentsline{toc}{chapter}{Irodalomjegyzék}

		\bibitem{test} 
		\href{http://example.com}{Example.com} (2019.04.06.)
	\end{thebibliography}
	
	\chapter*{Mellékletek}
	\pagenumbering{gobble}
	\addcontentsline{toc}{chapter}{Mellékletek}
	\section*{DVD tartalma}
	TODO (tartalom, mappaszerkezet)
\end{document}
