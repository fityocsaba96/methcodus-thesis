\UseRawInputEncoding
\documentclass{elteikthesis}

\title{Methcodus}
\date{2019}
\author{Fityó Csaba}
\degree{programtervező informatikus BSc}
\supervisor{Nagy Sára}
\affiliation{mesteroktató}
\extsupervisor{Czeller Ildikó}
\extaffiliation{adattudós}
\university{Eötvös Loránd Tudományegyetem}
\faculty{Informatikai Kar}
\department{Algoritmusok és Alkalmazásaik}
\departmentSecondLine{Tanszék}
\city{Budapest}
\logo{elte-crest}

\begin{document}

	\documentlang{magyar}

	\maketitle

	\tableofcontents

	\chapter{Bevezetés}

		\section{Motiváció}
		TODO (témaválasztás indoklása)

		\section{Feladatleírás}
		TODO (röviden, közérthetően)

	\chapter{Felhasználói dokumentáció}

		\section{Felhasználási igények}
		TODO (mire való, kik mikor mire használják, összehasonlítás Codewarssal és hagyományos fejlesztőkörnyezettel)
		
		\section{Használati követelmények}
		TODO (rendszerkövetelmények, tudásbeli követelmények, elérési link)
		
		\section{Alkalmazás funkciói}
		TODO (minden funkció leírása képernyőképekkel, külön regisztrált és nem regisztrált felhasználóra, külön helyes és hibás használat)
		
		\section{Szoftverfejlesztési módszerek}
		TODO (tdd, párprogramozás bemutatása)

	\chapter{Fejlesztői dokumentáció}

		\section{Probléma specifikáció}
		TODO (célok meghatározása fejlesztői szemmel)

		\section{Tervezés}

			\subsection{Használati esetek}
			TODO (use case diagram)
			
			\subsection{Felület terv}
			TODO (mockupok)
			
			\subsection{Architektúra terv}
			TODO (rövid leírás a technikai megvalósításról)
		
		\section{Szerver architektúra}

			\subsection{Authentikáció}
			TODO (folyamata)
			
			\subsection{Adatbázis}
			TODO (modellek leírása és ábra, kapcsolatok)
			
			\subsection{Végpontok}
			TODO (leírásuk, működésük, kérés és válasz tartalma, hibalehetőségek)
			
			\subsection{Felhasznált technológiák}
			TODO (mit tud, mire és hogy használom: rest, node, typescript, nest, mongodb, docker, ubuntu, java, javascript, junit, jasmine, babel, core-js, websocket, socket.io, passport, jwt, ...)
			
			\subsection{Kód felépítése}
			TODO (osztálydiagram, mappaszerkezet, rétegek)
		
		\section{Kliens architektúra}

			\subsection{Authentikáció}
			TODO (folyamata)
			
			\subsection{Képernyők}
			TODO (leírásuk, komponenseik)
			
			\subsection{Komponensek}
			TODO (leírásuk, működésük, hibalehetőségek)
			
			\subsection{Felhasznált technológiák}
			TODO (mit tud, mire és hogy használom: angular, typescript, redux, spa, emarsys design system, ...)
			
			\subsection{Kód felépítése}
			TODO (komponensdiagram, mappaszerkezet, rétegek)
		
		\section{Komponensek közti kommunikáció}
		TODO (szerver, kliens és adatbázis összekötés, szekvenciadiagram egy összetettebb funkcióra)

		\section{A tesztelő rendszer működése}
			Az alkalmazás leglényegesebb része a tesztelő rendszer. A felhasználók által megoldandó programozási feladatok megoldása közben, illetve azok beküldésekor nyújt támpontot számukra azáltal, hogy kiértékeli a megoldásukat és egységes formájú jelentést készít annak állapotáról. Akár tesztvezérelt fejlesztéssel oldja meg a felhasználó a feladatot, akár nem, ez a rendszer felel a megoldás teszteléséért. A tesztelő rendszer a szerveroldalon található, HTTP kapcsolaton keresztül folyik a kommunikáció. Megkapjuk a teszteléshez szükséges adatokat, elvégezzük a teszteket, utána pedig válaszolunk a teszteredményekkel.
			
			Miután megérkeznek HTTP kérésen keresztül a megfelelő adatok, elindítjuk a tesztelést külön folyamatban parancssori programként, a tesztelendő megoldás programozási nyelve szerinti, azon a nyelven írt tesztelő program indításával. Java esetén a \texttt{java}, JavaScript esetén pedig a \texttt{node} parancsot használjuk erre, a teszteléshez szükséges adatokat pedig környezeti változókon keresztül, szövegesen kapja meg a tesztelő. Miután a tesztelő elvégezte feladatát, tudatnia kell az eredményeket az őt hívó szülőfolyamattal. Sikeres tesztelés esetén kiírja a standard kimenetre a teszteredményeket. Hiba esetén kilépési kódokat használunk:
			\begin{itemize}
				\setlength\itemsep{-0.5em}
				\item[1:] a beküldött kódokban lévő szintaktikai hibát jelez
				\item[2:] a beküldött kódokban lévő fordítási hibát jelez
				\item[255:] egyéb, ismeretlen hiba
			\end{itemize}
			A tesztelő folyamat terminálásakor, a szülőfolyamatban kiolvassuk a standard kimenetet, illetve a kilépési kódot. Ezek alapján, illetve hiba esetén egy hibaüzenetet előállítva küldjük vissza a HTTP választ, ami alapján megjelenítjük a felhasználónak a megoldása állapotát. TODO (feluleten levo tesztelesi dolgokrol irni ide?)
		
			A tesztek futtatására két mód van. Az egyik módszer az egységteszt keretrendszer kódján alapuló tesztelés, másik pedig az egységes formátumon alapuló tesztelés. Mindkét módszer képes együttműködni Java és JavaScript nyelven írt kódokkal.

		\section{Fejlesztőkörnyezet}
		TODO (rendszerkövetelmények, beállítás, futtatás)
		
		\section{Fejlesztési folyamat}

			\subsection{Csomagkezelés}
			TODO (npm)
			
			\subsection{Verziókövetés}
			TODO (git)
			
			\subsection{Continuous Integration}
			TODO (codeship: lint, teszt, deploy)
			
			\subsection{Continuous Delivery}
			TODO (heroku)
		
		\section{Tesztelés}

			\subsection{Egységtesztek}
			TODO (lib leírás, tesztelés eredménye)
			
			\subsection{Manuális tesztek}
			TODO (minden funkció minden kimenetelére given-when-then leírás)
		
		\section{Továbbfejlesztési lehetőségek}
		TODO (clean code, élő menet, párprogramozásos élő menet)

	\begin{thebibliography}{99}

		\addcontentsline{toc}{chapter}{Irodalomjegyzék}

		\bibitem{test} 
		\href{http://example.com}{Example.com} (2019.04.06.)

	\end{thebibliography}
	
	\chapter*{Mellékletek}

	\pagenumbering{gobble}
	\addcontentsline{toc}{chapter}{Mellékletek}

	\section*{DVD tartalma}
	TODO (tartalom, mappaszerkezet)

\end{document}
